\documentclass{article}
\usepackage[T2A]{fontenc}
\usepackage[a5paper]{geometry}
\usepackage[utf8]{inputenc}
\usepackage[english, russian]{babel}
\usepackage{ragged2e}
\usepackage{fancyhdr}
\usepackage{setspace}
\usepackage{amssymb}
\usepackage{amsmath}
 
\setcounter{page}{170}
 
\renewcommand{\headrulewidth}{0pt}
\pagestyle{fancy}
\cfoot{\overline\mbox{{\quad\quad\textsl{\thepage}\quad\quad}}\raisebox{1mm}}
 
\begin{document}
\noindentи этот процесс последовательного вычисления присоединённых векторов продолжается до получения общей суммы собственных и присоединённых векторов, равной кратности рассматриваемого корня, т.е. \textit{m}. После этого, например, в случае \textit{одного независимого собственного вектора $H_1$} и результата вычисления \textit{присоединённых к нему векторов $H_2, H_3, H_4,...,$ соответствующая часть общего решения данной системы представляет собой следующую линейную комбинацию с произвольными постоянными} [28] : 
$$
    y = C_1H_1e^{\lambda_{1}x} + C_2\Bigg[\frac{x}{1!}H_1 + H_2\Bigg]e^{\lambda_{1}x} + C_3\Bigg[\frac{x^2}{2!}H_1 + \frac{x}{1!}H_2 + H_3\Bigg]e^{\lambda_{1}x} + \\ 
$$
$$
+ C_4\Bigg[\frac{x^3}{3!}H_1 + \frac{x^2}{2!}H_2 + \frac{x}{1!}H_3 + H_4\Bigg]e^{\lambda_{1}x}...
\eqno(35.6)
$$
и так далее до слагаемого, замыкающего $m$ слагаемых фундаментальной системы решений, соответствующих рассматриваемому кратному корню характерестического уравнения данной системы. 
 
Рассмотрим \textit{второй из методов решения данной системы} в случае указанного кратного корня. Он состоит в следующем. Сначала \textit{вычисляется вспомогательное число $k$} по формуле
$$
k = Rang(A-\lambda_{1}E) - (n-m),
$$
где $n$ - порядок системы; $m$ - кратность корня $\lambda_{1}$. Затем \textit{составляется выражение для части общего решения данной системы, соответствующего этому кратному корню, в виде}
$$
y = \big[H_0 + H_1x + H_2x^2 +...+ H_kx^k\big]e^{\lambda_{1}x},\eqno(35.7)
$$
где $H_0,H_1,H_2,...,H_k$ - числовые векторы $n$-го порядка с неопределёнными коэффицентами. \textit{Это выражение подставляется в систему} (35.1), и затем в результате подстановки и сокращения на $e^{\lambda_{1}x}$ приравниваются коэффиценты при одинаковых степенях переменной $x$. В итоге получается система линейных алгебраических уравнений относительно неопределённых коэффицентов векторов $H_0,H_1,H_2,...,H_k$. Далее \textit{определяем общее решение этой системы,} которое обязательно будет содержать $m$ произвольных постоянных. Затем подставляем полученные значения всех коэффицентов в (35.7) и \textit{разрешаем это выражение относительно  m произвольных постоянных. Результатом оказывается часть фундаментальной системы решений данного уравнения, соответствующая данной совокупности m кратных корней.}
 
\textbf{\textit{Пример.}} Решить систему уравнений $\begin{cases}x = 3x - y, \\ y = 4x - y. \end{cases}$
 
 
\textbf{\textit{Решение.}} Составим характеристическое уравнение
$$
\begin{vmatrix}
3-\lambda & -1 \\
4 & -1-\lambda
\end{vmatrix} = 0.
$$
 
Решая уравнение, определяем его корни:
$$
(3-\lambda)(-1-\lambda)+4=0; \lambda^2 - 2\lambda+1=0; \lambda_1=\lambda_2=1.
$$
 
Решим данную систему сначала первым из рассмотренных методов. Вычисляем собственный вектор:
$$
(A-1\cdot E)H_1=\begin{pmatrix} 
2 & -1 \\
4 & -2
\end{pmatrix} \begin{pmatrix}
    h1 \\
    h2
\end{pmatrix} = 0; \begin{cases}
    2h_1-h_2=0, \\
    4h_1-2h_2=0;
\end{cases}
$$
$$
2h_1-h_2=0; 2h_2=2; h_1=1; H_1=\begin{pmatrix}
    1 \\
    2
\end{pmatrix}.
$$
 
Теперь вычисляем присоединённый вектор:
$$
(A-E)H_2=H_1; \begin{pmatrix} 
2 & -1 \\
4 & -2
\end{pmatrix} \begin{pmatrix}
    h_1 \\
    h2
\end{pmatrix}=\begin{pmatrix}
    1 \\
    2
\end{pmatrix}; \begin{cases}
    2h_1-h_2=1, \\
    4h_1-2h_2=2.
\end{cases}
$$
 
Очевидно, условия \textit{условия теоремы Кронекера-Капелли} [10] выполнены и, по аналогии с предыдущим, имеем:
\end{document}
